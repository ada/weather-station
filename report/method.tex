\section{Method}
\ref{sec:method}
In order to meet the initial requirements, several sensors listed below were acquired and conntect to the arduino.
The functionality and usage of each sensor is described in the following subsections. An ATMega2560 microcontroller
has been adapted to acquire data from the sensor and the data is constructed into a packet. A ready packet is sent to the
ESP8266 wifi module via UART. The wifi module is consisted of a microcontroller which is programmed to connect to a WiFi hotspot and
do the authentication, recognizing the packets and check the validity of the received packets. On ESP8266 one TCP client sends packets to the NodeJs server
and one HTTP server displays information such as connection status and signal strenth of the module. In figure \ref{fig:arduino} a connection diagram
between these module is represented.

A locally hosted TCP server on a PC has been developed to get weather data from the connected ESP8266 clients. In addition to
storing the data in a cloud based database, a websocket server is run and broadcasts the latest received data to all the connect users.
The users access the latest availabe data through WebSocket but can also fetch, filter and visualize weather data from the stored entries in the database.
The TCP server and the database have support for unlimited number of weather stations which allows the user to select a specific weather station on the browser and
study the data. In figure \ref{fig:arduino} a connection representation of the server, database and the clients is visualized.


\begin{figure}[p]
    \centering
    \includegraphics[width=0.8\linewidth]{arduino}
    \caption{Arduino acquires data from the sensors periodically which are sent to the ESP8266 Wifi Module.
      The data is also available to be sent to the SIM800L module whenever requested.}
    \label{fig:arduino}
\end{figure}

\begin{figure}[p]
    \centering
    \includegraphics[width=0.8\linewidth]{server}
    \caption{Server structure.}
    \label{fig:arduino}
\end{figure}


\subsection{Power supply (PSU)}
The subsystems require 5V respective 3.3V. With all subsystem powered on it draws around 1A.
Power supply consist of LT1086 voltage regulator which is capable of delivering 1.5A.
Two PSU was made to supply 5V respective 3.3V.
\begin{figure}[p]
    \centering
    \includegraphics[width=0.8\linewidth]{voltage_regulator}
    \caption{Voltage regulator schematic to provide with 3.3v and 5.0 volts with maximum current of 1.5A.}
    \label{fig:voltage_regulator}
\end{figure}



\subsection{UART}
For sensors which are communicating between 3.3V and 5V TTL interfaces over UART a voltage divider has been implemented.
\begin{figure}[p]
    \centering
    \includegraphics[width=0.8\linewidth]{voltage_divider}
    \caption{Voltage divider schematic to adjust the voltage between 3.3V and 5.0V.}
    \label{fig:voltage_divider}
\end{figure}




\subsection{Packet}
Communication between Arduino, ESP8266, NodeJS server is done via custom packet of 12 Bytes data.
Since TCP checksum is only intended for the header, a custom checksum value has been added to the packet.
The overall packet consist of 2 bytes start flags, 1 bytes checksum, 1 bytes data kind, 4 bytes data value, 4 bytes timestamp.

\begin{figure}[p]
    \centering
    \includegraphics[width=\linewidth]{packet1}
    \caption{Custom packet structure is used for communication between Arduino, ESP8266 and the host server.}
    \label{fig:packet1}
\end{figure}


\subsection{Sensor MQ-4}
Measures LPG, Methane (CH4), H2, CO, Alcohol, Smoke.
Sensor is composed by micro AL2O3 ceramic tube, Tin Dioxide (SnO2) sensitive layer,
measuring electrode and heater are fixed into a crust made by plastic and stainless steel net.
\begin{figure}[p]
    \centering
    \includegraphics[width=0.5\linewidth]{MQ4}
    \caption{Graphical representation of MQ4 gas sensor.}
    \label{fig:MQ4}
\end{figure}


\subsection{Sensor MQ-2}
Measures LPG, Methane (CH4), H2., CO, Alcohol, Propane, Smoke.
Sensor is composed by micro AL2O3 ceramic tube, Tin Dioxide (SnO2) sensitive layer,
measuring electrode and heater are fixed into a crust made by plastic and stainless steel net.
\begin{figure}[p]
    \centering
    \includegraphics[width=0.5\linewidth]{MQ2}
    \caption{Graphical representation of MQ2 gas sensor.}
    \label{fig:MQ2}
\end{figure}

\subsection{Sensor MH-Z19 NDIR CO2}
MH-Z19 NDIR infrared gas module is a common type, small size sensor, using non-dispersive
infrared (NDIR) principle to detect the existence of CO 2 in the air, with good selectivity, non-oxygen
dependent and long life. Built-in temperature sensor can do temperature compensation; and it has
digital output and analog voltage output. It is developed by the tight integration of mature infrared absorbing
gas detection technology, precision optical circuit design and superior.

By sending the following bytes (0xFF, 0x01, 0x86, 0x00, 0x00, 0x00, 0x00, 0x00, 0x79) to the sensor via UART an response array of bytes consisting of
(Starting byte, command, High level, Low level, -, -, -, -, -, checksum) is received. Equation \ref{equ:gasConcentration} is used for convertion of the response to a PPM integer value.
The rest is of the return values is ignored.
\begin{equation}
  A_{Gas concentration} (PPM) = B_{high level} * 256 + C_{low level}
  \label{equ:gasConcentration}
\end{equation}



\begin{figure}[p]
    \centering
    \includegraphics[width=0.5\linewidth]{MH-Z19}
    \caption{Graphical representation of MH-Z19 CO2 sensor.}
    \label{fig:MH-Z19}
\end{figure}



\subsection{Sensor GY-NEO6MV2-GPS}
The NEO-6 module series is a family of stand-alone GPS receivers featuring the high performance u-blox 6
positioning engine. These flexible and cost effective receivers offer numerous connectivity options in a miniature
16 x 12.2 x 2.4 mm package. Their compact architecture and power and memory options make NEO-6 modules
ideal for battery operated mobile devices with very strict cost and space constraints.
The 50-channel u-blox 6 positioning engine boasts a Time-To-First-Fix (TTFF) of under 1 second. The dedicated
acquisition engine, with 2 million correlators, is capable of massive parallel time/frequency space searches,
enabling it to find satellites instantly. Innovative design and technology suppresses jamming sources and
mitigates multipath effects, giving NEO-6 GPS receivers excellent navigation performance even in the most
challenging environments.

\begin{figure}[p]
    \centering
    \includegraphics[width=0.5\linewidth]{GY-NEO6MV2-GPS}
    \caption{Graphical representation of GY-NEO6MV2 GPS sensor.}
    \label{fig:GY-NEO6MV2-GPS}
\end{figure}
