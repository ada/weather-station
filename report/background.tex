\section{Background and related work}
\label{sec:background}
Yash Mittal and et al. presented the development of an ‘Online Smart Weather
Station System’ for studying the correlation amongst multiple weather parameters
data collected over a period of 18 months. The study focuses on finding the
correlations between Temperature, Relative Humidity and Wind Speed, which are
the most important weather parameters for agriculture and other field activities.
Correlation among temperature, humidity and wind-speed is presented using the mean
and standard deviation. Few important observations are made such as relative humidity
reduces when the air temperature increases, and vice versa. Also, the wind
speed seems to increase as the air temperature increases and humidity reduces.
These inferences are drawn by studying the mean and standard deviation in
day-wise plots for the three weather parameters studied. Possible
applications of this study are in agriculture, construction and manufacturing \cite{7443621}.

In 2010 Z. Fang et al. presented a new portable micro weather station, which can sense temperature,
relative humidity, pressure and anemometer. They developed a small size multi-sensor chip system, display and a
power management system. A drag force wind sensor using the torque of cantilever to measure the velocity of wind is developed.
The wind direction can be measured by perpendicularly encapsulating the two wind sensor. All the results exhibit outstanding performances
of the micro weather station \cite{5592239}.

In 2014 L. Yao et al. presented a study on WebSocket-based technology to build real-time meteorological wireless sensor network
information publishing platform. They analyzed the principles and problems of the existing Web real-time communication technologies and put
forward the real-time meteorological information publishing platform (RT-MIPP) solution based on WebSocket. At the same time, through the
comparison of all kinds of real-time Web communication technology, the WebSocket significant advantages in low-latency and low network
throughput is obtained. Their results show that, using .NET and WebSocket technology, greatly improve the performance of system in need of real-time
communication \cite{6975834}.
