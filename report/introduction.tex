\section{Introduction}
\label{sec:introduction}
Like all other living organisms we are part of a larger system and in the case of
humans, we are dependent on our environment. Our environment is continuously in change
both in short but also in long timespans and weather often varies greatly from
place to place within a short distance apart. Previous research found that
distributed ground weather station outperform satellite-based observations
in measuring rainfall and soil conditions at different depth which are greatly
emphasized in agriculture \cite{Mendelsohn2007}.

Weather monitoring and prediction are important tools to provides us with necessary information for the
infrastructures such as precision agriculture \cite{6878963}, intelligent homes,
military, transportation, healthcare, disaster prevention, construction and manufacturing to work correctly.
However, currently available devices are expensive and many don't offer wireless capabilities.

This paper discusses the system architecture and implementation of a distributed cloud-based weather monitoring system with wireless functionality and
real-time accessibility using a client browser. The weather station is aimed at typical home user and for research purposes. The proposed system monitors environmental parameters including
temperature, air humidity, barometric pressure, carbon dioxide, wind speed and direction, precipitation amount
and rate, methane, butane, time and location where the latest provides with planetary lunar and solar times such
as lunar phase, sunrise and sunset times but also spatial information such as distance to moon.

Furthermore, the proposed system provides data acquisition capacity from distributed WSN (Wireless Sensor Network) stations,
replication of data to reliable database and allow online user access to weather data in real-time.
An embedded computer system based server is developed to acquire data from wirelessly connected weather
stations as well as periodically push latest sensor data to the central cloud server.

The paper is organized as follows. In section \ref{sec:background},
three related works by others are presented. Section \ref{sec:method} describes the electronics, tools and the design decision for the hardware and communication layer. The server and client application design decisions are presented in section \ref{sec:method}.
The data generated by a weather station node and its storage are discussed in section \ref{sec:results}.
In \ref{sec:conclusions} a brief discussion is made on the observation and the results along the the scope for future work.
