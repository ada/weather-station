\section{Conclusions and Future Directions}
\label{sec:conclusions}
This paper has described the development of a cloud-based remote environmental
monitoring system with embedded distributed weather stations and real-time accessibility.
We have provided an overview of the necessary tools and knowledge required
for engineering a distributed weather station.

The developed system met all except two of the requirements stated at the project proposal in section \ref{sec:introduction}.
Those requirements which were not completed consist of wind speed, wind direction and rain amount measurement which
were postponed due to cost and timing constraints. It's worth to mention that when new sensors such as wind speed are available,
it's a matter of minutes to attach the module and restart the system.

A natural evolution of this distributed weather station would be to design and build sensors with superior
accuracy and operating conditions compared to the current off-the-shelf ones. With higher reliability,
this system could be used for private home control systems, industrial, agricultural, commercial or research purposes such as integration of
data from multiple weather stations to develop an experimental artificial intelligent system
with weather prediction capabilities. 
